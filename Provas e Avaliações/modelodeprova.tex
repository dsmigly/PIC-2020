%http://www.math.uconn.edu/~kconrad/blurbs/
\documentclass[oneside,a4paper,12pt]{article}
%\usepackage[english,brazilian]{babel}
\usepackage[alf]{abntex2cite}
\usepackage[utf8]{inputenc}
\usepackage[T1]{fontenc}
\usepackage[top=20mm, bottom=20mm, left=20mm, right=20mm]{geometry}
\usepackage{framed}
\usepackage{booktabs}
\usepackage{color}
\usepackage{hyperref}
\usepackage{graphicx}
\usepackage{float}
\usepackage{pstricks}
\graphicspath{{./Figuras/}}    
\definecolor{shadecolor}{rgb}{0.8,0.8,0.8}

\usepackage{tikz}
\usepackage[utf8]{inputenc}
\usepackage{mathtext}
\usepackage{graphicx}
\usepackage{wrapfig}
\usepackage[T1]{fontenc}
\usepackage{blindtext}
\usepackage{tasks}
\usepackage{fancybox}
\usepackage{amsthm}
\usepackage{setspace}
\usepackage{amsmath}
\usepackage{amsfonts}
\usepackage{amssymb}
\usepackage{graphicx, color}
\newcommand{\sen}{{\rm sen}}
\newcommand{\tg}{{\rm tg}}
\newcommand{\cotg}{{\rm cotg}}
\newcommand{\cossec}{{\rm cossec}}
\newcommand{\arctg}{{\rm arctg}}
\newcommand{\arcsen}{{\rm arcsen}}
\newcommand{\negrito}[1]{\mbox{\boldmath{$#1$}}} 
\usepackage{pifont}
\usepackage[framemethod=TikZ]{mdframed}
\newcommand{\heart}{\ensuremath\heartsuit}
\newcommand{\diamonde}{\ensuremath\diamondsuit}
\theoremstyle{Colorido}
\newtheorem{questao}{\textcolor{Floresta}{\textit{\bf Questão}}}
\newtheoremstyle{Colorido}{}{}{\color{Floresta}}{}{\color{Floresta}\bfseries}{}{ }{}
\newtheorem{theorem}{Theorem}
\newtheoremstyle{solu}{}{}{}{}{\color{red}\bfseries}{}{ }{}
\theoremstyle{solu}
\newtheorem*{resp}{Solução}
\newtheoremstyle{dotlessP}{}{}{}{}{\color{Floresta}\bfseries}{}{ }{}
\theoremstyle{dotlessP}
\newtheorem{sol}{Questão}
%FAZ EDICOES AQUI (somente no conteudo que esta entre entre as ultimas  chaves de cada linha!!!)
\newcommand{\universidade}{Programa de Iniciação Científica OBMEP}
\newcommand{\centro}{PIC 2018}
%\newcommand{\departamento}{Departamento}
%\newcommand{\curso}{Curso}
\newcommand{\professor}{Douglas de Araujo Smigly}
\newcommand{\disciplina}{Tópicos de Geometria Olímpica}
\newcommand{\entrega}{5 de maio de 2018}
\DeclareSymbolFont{extraup}{U}{zavm}{m}{n}
\DeclareMathSymbol{\varheart}{\mathalpha}{extraup}{86}
\DeclareMathSymbol{\vardiamond}{\mathalpha}{extraup}{87}
	\cornersize{.3} 
	\mdfdefinestyle{MyFrame}{%
    linecolor=blue,
    outerlinewidth=2pt,
    roundcorner=20pt,
    innertopmargin=\baselineskip,
    innerbottommargin=\baselineskip,
    innerrightmargin=20pt,
    innerleftmargin=20pt,
    backgroundcolor=gray!24!white}
%ATE AQUI !!!

\begin{document}
\definecolor{Floresta}{rgb}{0.13,0.54,0.13}
	\pagestyle{empty}
	
	\begin{center}
	\includegraphics[width=\linewidth/3]{logo_pic}%LOGOTIPO DA INSTITUICAO
	 	\vspace{0pt}
	 	
		\universidade
		\par
		\centro
		\par
%		\departamento
		\par
%		\curso
		\par
		\vspace{24pt}
		\LARGE \textbf{Avalia\c c\~ao - Ciclo I}
		
	\end{center}
	
	\vspace{24pt}
	
	%
%	\begin{tabular}{ |l|p{12cm}| }
%		
%		\hline
%		\multicolumn{2}{|c|}{\textbf{Dados de Identificação}} \\
%			\hline
%		Disciplina:        &    \disciplina          \\
%		\hline
%		Professor:         &    \professor           \\
%	\hline
%	Aluno(a):         &\\
%		\hline
%	Multiplicidade:  & \ \ \ \ \ \ \vline Nível: \vline\\
%	
%		\hline
	%\end{tabular}
	%
	\begin{tabular}{ |l|p{12cm}| }
		
		\hline
		\multicolumn{2}{|c|}{\textbf{Dados de Identificação}} \\
			\hline
		Curso:        &  \disciplina \\
			\hline
		Nome:        &  \\
		\hline
		Nível:      &  \\
		\hline
				Multiplicidade:      &  \\
		\hline
				Assinatura:      &  \\
		\hline
				Data de entrega:      &  \entrega \\
		\hline
	\end{tabular}
	
	\vspace{24pt}
	\vspace{40pt}
	\begin{center}
		\begin{tabular}{ |c|p{1.2cm}| }
		
		\hline
		\multicolumn{2}{|c|}{\textbf{Pontuação}} \\
			\hline
		\centering\textbf{Questão}        &  \textbf{Nota}\\
		\hline
		Questão 1        &\\
		\hline
		Questão 2        &\\
		\hline
		Questão 3        &\\
		\hline
		Questão 4        &\\
		\hline
		\textbf{Total}        &\\
		\hline
	\end{tabular}
	\end{center}
	\vspace{24pt}
	
	%\begin{snugshade}
	%	\section{O... aumento  }  
	%\end{snugshade}
	\newpage	
	\begin{sol}
\textit{(2,5 pontos)} 

\begin{tasks}[counter-format={(tsk[a])},label-width=3.6ex, label-format = {\bfseries}, column-sep = {20pt}](1)
\task[\textcolor{blue}{$\negrito{(a)} $}] Encontre as médias aritmética, geométrica, harmônica e quadrática dos números $5$ e $8$.
\task[\textcolor{blue}{$\negrito{(b)} $}] Faça um esboço representando geometricamente as medidas obtidas no item anterior a partir de dois segmentos, descrevendo como a construção foi feita.
\end{tasks}
\end{sol}
		%\vspace{60pt}
		\newpage	
	\begin{sol}
\textit{(2,5 pontos)} \newline \newline
Sejam $\alpha, \beta$ e $\gamma$ números reais positivos não-nulos (isto é, $\alpha, \beta, \gamma \in \mathbb{R}^{*}_{+}$) tais que $\alpha + \beta + \gamma = 1.$ Demonstre a validade das seguintes desigualdades, e mostre que a igualdade de ambas ocorre se, e somente se $\alpha = \beta = \gamma = \frac{1}{3}$:
\begin{tasks}[counter-format={(tsk[a])},label-width=3.6ex, label-format = {\bfseries}, column-sep = {20pt}](2)
\task[\textcolor{blue}{$\negrito{(a)} $}] $\frac{\alpha \beta}{\gamma} + \frac{\beta \gamma}{\alpha} + \frac{\alpha \gamma}{\beta} \ge 1;$
\task[\textcolor{blue}{$\negrito{(b)} $}] $\frac{\alpha^2 + \beta^2}{\gamma} + \frac{\beta^2 + \gamma^2}{\alpha} + \frac{\alpha^2 + \gamma^2}{\beta} \ge 2.$
\end{tasks}
\end{sol}

\newpage
	\begin{sol}
\textit{(2,5 pontos)} \newline \newline Sejam $a$, $b$ e $c$ as medidas dos lados de um triângulo, e $\alpha, \beta$ e $\gamma$ os ângulos respectivos deste triângulo em radianos. Então, a seguinte desigualdade é verdadeira:
\[ \frac{\pi}{3} \le \frac{a\alpha + b\beta + c\gamma}{a + b + c} < \frac{\pi}{2} \]

\begin{tasks}[counter-format={(tsk[a])},label-width=3.6ex, label-format = {\bfseries}, column-sep = {20pt}](1)
\task[\textcolor{blue}{$\negrito{(a)} $}] Considere o triângulo $ABC$ abaixo. Utilizando a desigualdade descrita acima, qual é o menor valor possível para $\overline{AC} + \overline{BC}?$
\begin{center}
\includegraphics[scale=0.38]{triangulo_p1}
\end{center}
\task[\textcolor{blue}{$\negrito{(b)} $}] Prove que a igualdade 
\[  \frac{\pi}{3} = \frac{a\alpha + b\beta + c\gamma}{a + b + c} \] ocorre se, e somente se, o triângulo for equilátero. 
\end{tasks}
\end{sol}
\newpage
	\begin{sol}
\textit{(2,5 pontos)} \newline \newline Considere o paralelepípedo abaixo de dimensões $a,b$ e $c.$ Prove que entre todos os paralelepípedos de área total fixada, o de maior volume é o cubo.
\begin{center}
\includegraphics[scale=0.48]{paralelepipedo_p1}
\end{center}
% Resp: i e i-4
\end{sol}



		%\vspace{60pt}
%		\newpage
		%QUESTAO 2
		
		%\vspace{60pt}
	
		%QUESTAO 3

		%\vspace{60pt}
		
		%QUESTAO 4 (OBJETIVA)

		
		%a)(  ) Alternativa A.
		
	%	b)(  ) Alternativa B.
		
	%	c)(  ) Alternativa C.
		
	%	d)(  ) Alternativa D.
		
	\flushbottom
	\flushright
%	"Alguma frase bonita de fim de prova"\\(autor da frase bonita)
\end{document}
