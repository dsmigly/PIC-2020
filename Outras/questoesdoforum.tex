%http://math.mit.edu/~drew/
%https://math.stackexchange.com/questions/93689/software-for-galois-theory
%SetClassGroupBounds("GRH"); 
%K := QuadraticField(9);
%ClassNumber(K);

%https://en.wikipedia.org/wiki/List_of_triangle_inequalities

\documentclass[12pt, a4paper]{article}
\usepackage[bottom=2cm,top=3cm,left=3cm,right=2cm]{geometry}
\usepackage[utf8]{inputenc}
\usepackage{CJKutf8}
\usepackage{mathtext}
\usepackage{graphicx}
\usepackage{wrapfig}
\usepackage[T1]{fontenc}
\usepackage{blindtext}
\usepackage{tasks}
\usepackage{setspace}
\usepackage{amsmath}
\usepackage{amsfonts}
\usepackage{amssymb}
\usepackage{ wasysym }
\usepackage[portuguese]{babel}
\usepackage[utf8]{inputenc}
\usepackage{mathtext}
\usepackage{graphicx}
\usepackage{wrapfig}
\usepackage[T1]{fontenc}
\usepackage{blindtext}
\usepackage{tasks}
\usepackage{setspace}
\usepackage{amsmath}
%\usepackage{geometry}
\usepackage{amsthm}
\usepackage{graphics}
%\usepackage{amsfonts}
%\usepackage{lipsum}
\usepackage{amssymb}
\usepackage{CJKutf8} %Pacote para escrever em japonês \begin{CJK}{UTF8}{min} \end{CJK}
\usepackage[portuguese]{babel}
\usepackage{multicol}
% \usepackage{colorspace}
\usepackage{graphicx, color}
\newcommand{\mdc}{{\rm mdc}}
\newcommand{\sen}{{\rm sen}}
\newcommand{\tg}{{\rm tg}}
\newcommand{\cotg}{{\rm cotg}}
\newcommand{\cossec}{{\rm cossec}}
\newcommand{\arctg}{{\rm arctg}}
\newcommand{\arcsen}{{\rm arcsen}}
\newcommand{\pulaquestao}{\newline\newline}
\newcommand{\negrito}[1]{\mbox{\boldmath{$#1$}}} 
\usepackage{pifont}
\newcommand{\heart}{\ensuremath\heartsuit}
\newcommand{\diamonde}{\ensuremath\diamondsuit}
\newtheorem{defi}{Definição}
\newtheorem{propo}{Proposição}
\newtheorem{dem}{Demonstração}
\newtheorem{coro}{Corolário}
\DeclareSymbolFont{extraup}{U}{zavm}{m}{n}
\DeclareMathSymbol{\varheart}{\mathalpha}{extraup}{86}
\DeclareMathSymbol{\vardiamond}{\mathalpha}{extraup}{87}
\setlength{\parindent}{0pt}
\usepackage[framemethod=TikZ]{mdframed}
%\usepackage{lipsum}
\mdfdefinestyle{MyFrame}{%
    linecolor=blue,
    outerlinewidth=2pt,
    roundcorner=20pt,
    innertopmargin=\baselineskip,
    innerbottommargin=\baselineskip,
    innerrightmargin=20pt,
    innerleftmargin=20pt,
    backgroundcolor=white!50!white}
    
%\mdfdefinestyle{Solução}{%
%    linecolor=blue,
%    outerlinewidth=1pt,
%    roundcorner=8pt,
%    innertopmargin=4pt%\baselineskip,
%    innerbottommargin=0pt%\baselineskip,
%    innerrightmargin=20pt,
%    innerleftmargin=20pt,
%    backgroundcolor=white!50!white}
    
    
    \mdfdefinestyle{DAS}{%
    linecolor=blue,
    outerlinewidth=2pt,
    roundcorner=20pt,
    innertopmargin=\baselineskip,
    innerbottommargin=\baselineskip,
    innerrightmargin=20pt,
    innerleftmargin=20pt,
    backgroundcolor=white!50!green}
% \definespotcolor{mygreen}{PANTONE 7716 C}{.83, 0, .00, .51}
% \definespotcolor{tuti}{}{0.6, 0, 1, .508}
\title{PIC}
\author{Douglas de Araujo Smigly}
\date{18 de abril de 2020}
\begin{document}
\definecolor{Floresta}{rgb}{0.13,0.54,0.13}
\maketitle
\begin{center}
\large\textbf{\textcolor{Floresta}{Questões do Fórum do Hotel de Hilbert - Ciclo I }}\\
\end{center}
%\begin{multicols*}{2}
%\setlength{\columnseprule}{0.78pt}
%\raggedcolumns
%\columnbreak
\textcolor{blue}{\bf(1)} Se $a - b = 1$ e $ab = 1,$ qual é o valor de $a^2 + b^2?$
\begin{figure}[!h]
    \centering
    \includegraphics{lgogo.jpg}
    \caption{Caption}
    \label{fig:my_label}
\end{figure}



%\newline\newline
\textcolor{blue}{\bf(2)} Na figura abaixo, $x$ é a média aritmética dos números que estão nos quatro círculos claros e $y$ é a média aritmética dos números que estão nos quatro círculos escuros. Quais são os valores de $x$ e $y$?
\textcolor{blue}{\bf(3)} Calcule:
\begin{tasks}[counter-format={(tsk[a])},label-width=3.6ex, label-format = {\bfseries}, column-sep = {0pt}](4)
\task[\textcolor{Floresta}{$\negrito{(a)} $}] $\varphi(7)$
\task[\textcolor{Floresta}{$\negrito{(b)} $}] $\varphi(9)$
\task[\textcolor{Floresta}{$\negrito{(c)} $}] $\varphi(16)$
\task[\textcolor{Floresta}{$\negrito{(d)} $}] $\varphi(30)$
\end{tasks}
\textcolor{blue}{\bf(4)} Sejam $a, n$ inteiros tais que $\mdc(a,n) = \mdc(a-1, n) = 1.$ Prove que
\[ \sum\limits_{k = 0}^{\varphi(n)-1} a^k \equiv 0 \pmod n\]
\textcolor{blue}{\bf(5)} Mostre que $18! \equiv -1 \pmod{437}.$
\newline
\newline
\textcolor{blue}{\bf(6)} (França TST 2012) Sejaa $n$ um inteiro positivo. Prove que se $\phi (n) \mid n-1$ e $n$tem no máximo $3$ divisores primos, então $n$ é um número primo.
\newline\newline
%https://artofproblemsolving.com/community/c6h473647p2651848
\textcolor{blue}{\bf(7)} Determine a soma de todos os valores de $n$ tais que $1 < n < 100$ e $\varphi(n) \mid n.$
\begin{tasks}[counter-format={(tsk[a])},label-width=3.6ex, label-format = {\bfseries}, column-sep = {0pt}](4)
\task[\textcolor{Floresta}{$\negrito{(a)} $}] $328$
\task[\textcolor{Floresta}{$\negrito{(b)} $}] $492$
\task[\textcolor{Floresta}{$\negrito{(c)} $}] $496$
\task[\textcolor{Floresta}{$\negrito{(d)} $}] $512$
\end{tasks}
%https://artofproblemsolving.com/community/c204707h1582278p9770630
\textcolor{blue}{\bf(8)} Encontre os últimos três dígitos de $7^{899}.$
%https://artofproblemsolving.com/community/q1h1197179p5868744
\newline
\newline
\textcolor{blue}{\bf(9)} Demonstre que, para $n\in \mathbb{Z}_+$ \[ \sum\limits_{k=1}^n \left\lfloor\frac{n}{k}\right\rfloor\varphi(k) = \frac{n(n + 1)}{2}\]
%https://artofproblemsolving.com/community/q1h1639547p10325316
\textcolor{blue}{\bf(10)} (Sérvia 2011) Seja $ n $ um número positivo ímpar tal que ambos $ \varphi(n+1) $ e $ \varphi(n)$ são potências de $2.$ Prove que $n + 1$ é potência de $2$ ou $n = 5.$
%https://artofproblemsolving.com/community/q1h597257p3544135
\newline
\newline
\textcolor{blue}{\bf(11)} Seja $n\geq 2$ um inteiro positivo. Prove que existem inteiros positivos  \[ a_1< a_2< \dots< a_n\] tais que \[ \varphi(a_1)>\varphi(a_2)>\dots>\varphi(a_n)\].
%q1h1471535p8540123
\textcolor{blue}{\bf(12)} Para um inteiro positivo $n$, prove que: $$\sum_{d|n} \cdot \left(-1\right)^{n/d}\cdot\phi(d)={0\ \ \ \ \ \ \text{se n é par} \brace -n \ \ \ \text{se n é ímpar}} $$
$\phi(d)$ é a Função de Euler.
\newline \newline
%https://artofproblemsolving.com/community/q1h1362534p7471162
\textcolor{blue}{\bf(13)} Sabe-se que, para todo inteiro positivo $n > 1,$ se $p_1, p_2, \ldots, p_k$ são números primos distintos que dividem $n,$ então
\[ \varphi(n) = n \prod\limits_{n = 1}^k \left(1 - \frac{1}{p_n}\right) = \]\[n \left(1 - \frac{1}{p_1} \right) \cdot \ldots \cdot \left(1 - \frac{1}{p_k} \right) \]
\begin{tasks}[counter-format={(tsk[a])},label-width=3.6ex, label-format = {\bfseries}, column-sep = {0pt}](1)
\task[\textcolor{Floresta}{$\negrito{(a)} $}] Utilize a fórmula acima para calcular $\varphi(15).$
\task[\textcolor{Floresta}{$\negrito{(b)} $}] Encontre o valor de $\varphi(2018).$
\task[\textcolor{Floresta}{$\negrito{(c)} $}] Encontre uma fórmula fechada para $\varphi$ quando $n$ é um número de quatro divisores.
\end{tasks}
\textcolor{blue}{\bf(14)} Seja $\varphi $ a função de Euler. Mostre que
\[ \varphi(n) = n \prod_{\substack{p \ \mbox{primo} \\ p \mid n }}\left( 1 - \frac{1}{p} \right) \]
\textcolor{blue}{\bf(15)} Mostre que, se $a \mid b,$ então $\varphi (a) \mid \varphi (b)$.
\newline
\newline
\textcolor{blue}{\bf(16)} Prove que nenhum número na sequência 
\[ 11, 111, 1111, \ldots, \underbrace{11 \ldots 111}_{n \ \mbox{vezes}} \]

É um quadrado perfeito.
\newline
\newline
\textcolor{blue}{\bf(17)} Se $1 \le n \le 1000$, mostre que $210$ é o menor valor inteiro que minimiza 
\[ \frac{\varphi(n)}{n}. \]
%https://brilliant.org/problems/from-login/minimum-totient-quotient/no-input/?topic_tag=numbertheory#_=_
%https://pt.wikipedia.org/wiki/Matriz_de_transi%C3%A7%C3%A3o
%https://pt.wikipedia.org/wiki/Matriz_de_adjac%C3%AAncia
\textcolor{blue}{\bf(18)} Encontre a quantidade de números inteiros positivos menores ou iguais a $168$ tais que $\mdc(n, 168) = 8.$
\newline\newline
\textcolor{blue}{\bf(19)} Gabriel escreveu no quadro as seguintes frações:
\[ \frac{1}{96}, \frac{2}{96}, \frac{3}{96}, \ldots , \frac{95}{96}, \frac{96}{96}. \]
Quantas dessas frações são irredutíveis?
\newline\newline
\textcolor{blue}{\bf(20)} Um \emph{número automórfico} pe definido como um inteiro positivo tal que os último dígitos de $n^m$, onde $m$ é um inteiro positivo, formam o próprio número $n.$ Por exemplo, $249$ é automórfico, pois
\[ 249^3 = 15438\textcolor{red}{249} \]
Vamos definir um número \emph{quasi-automórfico} como um número que aparece nos últimos dígitos de $n^m$ apenas quando $m$ é ímpar. Quantos são os números quasi-automórifcos menores que $1000?$
%https://brilliant.org/practice/eulers-theorem-level-5-challenges/?p=5
%https://brilliant.org/wiki/eulers-totient-function/
\newline\newline
\textcolor{blue}{\bf(21)} Se $n \ge 2$ e $n \in \mathbb{N},$ prove que 
\[ \frac{\sqrt{n}}{2} \le \varphi(n) \le n - 1 \]

\textcolor{blue}{\bf(22)} Considere o conjunto $\varphi^{-1}(m),$ ou seja:
\[ \varphi^{-1}(m) = \{ n : \varphi(n) = m \} \]
Calcule $\varphi^{-1}(36).$
\newline\newline
\textcolor{blue}{\bf(23)} Quais das expressões abaixo têm o mesmo valor de $\varphi(300)?$
\begin{tasks}[counter-format={(tsk[a])},label-width=3.6ex, label-format = {\bfseries}, column-sep = {0pt}](1)
\task[\textcolor{Floresta}{$\negrito{(a)} $}] $\varphi(3) \cdot \varphi(100)$
\task[\textcolor{Floresta}{$\negrito{(b)} $}] $\varphi(10) \cdot \varphi(30)$
\task[\textcolor{Floresta}{$\negrito{(c)} $}] $\left(\varphi(2) \right)^2 \cdot \varphi(3) \cdot \varphi(25)$
\task[\textcolor{Floresta}{$\negrito{(d)} $}] $\varphi(4) \cdot \varphi(3) \cdot \varphi(25)$
\task[\textcolor{Floresta}{$\negrito{(e)} $}] $\varphi(150)$
\end{tasks}
\textcolor{blue}{\bf(24)} Prove que , se $p$ é um número primo e $k \in \mathbb{N}^*,$ então 
\[ \phi(p^k) = p^k - p^{k-1}\]
%http://www.insa.nic.in/writereaddata/UpLoadedFiles/IJPAM/20005a81_22.pdf
%\end{multicols*}
\end{document}
SetClassGroupBounds("GRH"); 
K := QuadraticField(9);
ClassNumber(K);
Q := PolynomialRing(GF(2), 2);
Q;
K := QuadraticField();
G := GaloisGroup(K);
G;
\begin{CJK}{UTF8}{min}
露の世は 露の世ながら さりながら当時では老人と呼べる50歳代半ばでようやく授かったわが子への愛とその突然死を伝えた「露の世」のくだりは、その日記体句文集「おらが春」のクライマックスとなっています。

5月には数え二歳の誕生を迎えて詠んだ句に、
「這へ笑へ二つになるぞけさからは」と喜びを謳歌したばかり。

それが、翌6月にはもはや草葉の陰へと、その露の朝日に立ちどころに消えるごとく、儚くも身罷ってしまったとは。

人の世は朝露の如く無常なのだと、悔みを述べ慰問するあの人、この人。さは「さりながら」…それは確かにそうなのだけれども。
いかに「あきらめ顔しても、思い切りがたきは、恩愛のきづな也けり。」と、わが心中は耐えきれず切々と泣き崩れるばかり。わが子「さと」女を思う「大切」はやがて「あなた任せ」の境地へと通じていくものでしょう。
\end{CJK}