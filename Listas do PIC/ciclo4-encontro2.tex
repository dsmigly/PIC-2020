%https://pt.quora.com/Como-encontrar-todos-os-n%C3%BAmeros-inteiros-positivos-a-e-b-de-modo-que-ab-seja-divis%C3%ADvel-por-a-b

%https://math.stackexchange.com/questions/2775233/does-dn-always-depend-on-gcdn-sigman-when-sigman-anb-n-xn

%https://math.stackexchange.com/questions/3073053/if-sigman-2n-d-and-d-mid-n-is-it-true-that-d-gcdn-sigman?rq=1



%http://math.mit.edu/~drew/
%https://math.stackexchange.com/questions/93689/software-for-galois-theory
%SetClassGroupBounds("GRH"); 
%K := QuadraticField(9);
%ClassNumber(K);

%https://en.wikipedia.org/wiki/List_of_triangle_inequalities

\documentclass[12pt, a4paper]{article}
\usepackage[bottom=2cm,top=3cm,left=3cm,right=2cm]{geometry}
\usepackage[utf8]{inputenc}
\usepackage{CJKutf8}
\usepackage{mathtext}
\usepackage{graphicx}
\usepackage{wrapfig}
\usepackage[T1]{fontenc}
\usepackage{blindtext}
\usepackage{tasks}
\usepackage{setspace}
\usepackage{amsmath}
\usepackage{amsfonts}
\usepackage{amssymb}
\usepackage{ wasysym }
\usepackage[portuguese]{babel}
\usepackage[utf8]{inputenc}
\usepackage{mathtext}
\usepackage{graphicx}
\usepackage{wrapfig}
\usepackage[T1]{fontenc}
\usepackage{blindtext}
\usepackage{setspace}
\usepackage{amsmath}
%\usepackage{geometry}
\usepackage{amsthm}
\usepackage{graphics}
%\usepackage{amsfonts}
%\usepackage{lipsum}
\usepackage{amssymb}
\usepackage{CJKutf8} %Pacote para escrever em japonês \begin{CJK}{UTF8}{min} \end{CJK}
\usepackage[portuguese]{babel}
\usepackage{multicol}
% \usepackage{colorspace}
\usepackage{graphicx, color}
\newcommand{\mdc}{{\rm mdc}}
\newcommand{\mmc}{{\rm mmc}}
\newcommand{\sen}{{\rm sen}}
\newcommand{\tg}{{\rm tg}}
\newcommand{\cotg}{{\rm cotg}}
\newcommand{\cossec}{{\rm cossec}}
\newcommand{\arctg}{{\rm arctg}}
\newcommand{\arcsen}{{\rm arcsen}}
\newcommand{\pulaquestao}{\newline\newline}
\newcommand{\negrito}[1]{\mbox{\boldmath{$#1$}}} 
\usepackage{pifont}
\newcommand{\heart}{\ensuremath\heartsuit}
\newcommand{\diamonde}{\ensuremath\diamondsuit}
\newtheorem{defi}{Definição}
\newtheorem{propo}{Proposição}
\newtheorem{dem}{Demonstração}
\newtheorem{coro}{Corolário}
\DeclareSymbolFont{extraup}{U}{zavm}{m}{n}
\DeclareMathSymbol{\varheart}{\mathalpha}{extraup}{86}
\DeclareMathSymbol{\vardiamond}{\mathalpha}{extraup}{87}
\setlength{\parindent}{0pt}
\usepackage[framemethod=TikZ]{mdframed}
%\usepackage{lipsum}
\mdfdefinestyle{MyFrame}{%
    linecolor=blue,
    outerlinewidth=2pt,
    roundcorner=20pt,
    innertopmargin=\baselineskip,
    innerbottommargin=\baselineskip,
    innerrightmargin=20pt,
    innerleftmargin=20pt,
    backgroundcolor=white!50!white}
    
%\mdfdefinestyle{Solução}{%
%    linecolor=blue,
%    outerlinewidth=1pt,
%    roundcorner=8pt,
%    innertopmargin=4pt%\baselineskip,
%    innerbottommargin=0pt%\baselineskip,
%    innerrightmargin=20pt,
%    innerleftmargin=20pt,
%    backgroundcolor=white!50!white}
    
    
    \mdfdefinestyle{DAS}{%
    linecolor=blue,
    outerlinewidth=2pt,
    roundcorner=20pt,
    innertopmargin=\baselineskip,
    innerbottommargin=\baselineskip,
    innerrightmargin=20pt,
    innerleftmargin=20pt,
    backgroundcolor=white!50!green}
% \definespotcolor{mygreen}{PANTONE 7716 C}{.83, 0, .00, .51}
% \definespotcolor{tuti}{}{0.6, 0, 1, .508}
\title{PIC - Programa de Iniciação Científica}
\author{Douglas de Araujo Smigly}
\date{12 de setembro de 2020}
\begin{document}
\definecolor{Floresta}{rgb}{0.13,0.54,0.13}
\maketitle
\begin{center}
\large\textbf{\textcolor{Floresta}{Ciclo 4 - Encontro 2 - Aritmética}}\\
\end{center}
%\begin{multicols*}{2}
%\setlength{\columnseprule}{0.78pt}
%\raggedcolumns
%\columnbreak
\textcolor{blue}{\bf(1)} Use o Algoritmo de Euclides para calcular $\mdc(372,162),$ e use-o para escrever $\mdc(372,162) = 372x + 162y$ para algum inteiro $x$ e algum inteiro $y.$
(Obs.: o procedimento usado para expressar $\mdc(372,162),$ como $\mdc(372,162) = 372x + 162y$ pode ser realizado de maneira análoga para quaisquer dois inteiros não
ambos nulos $x$ e $y$ de forma que, dados inteiros $a$ e $b$ não ambos nulo, existem inteiros $x$ e $y$ tais que $\mdc(a,b) = ax + by.$ A igualdade é conhecida como \textit{Relação de Bézout}).
\newline\newline
\textcolor{blue}{\bf(2)} Sejam $a$, $b$ e $c$ números inteiros tais que $a$ divide $bc$ e $\mdc(a,b) = 1.$ Prove que $a$ divide $c.$
\newline\newline
\textcolor{blue}{\bf(3)} Sejam $a$, $b$ e $c$ números inteiros tais que $\mdc(a,b) = 1.$ Prove que $\mdc(a,bc) = \mdc(a,c).$
\newline\newline
\textcolor{blue}{\bf(4)} (Banco de Questões - OBMEP 2010) Mostre que existe um número natural $n$ tal que
\[
\underbrace{20182018 \dots 20182018}_{n\text{ vezes}}
\]
é múltiplo de $2019.$
\newline\newline
\textcolor{blue}{\bf(5)} (Banco de Questões - OBMEP 2017 - originalmente proposto na OBM 2011) Quantos são os pares ordenados $(a,b)$ com $a$ e $b$ inteiros positivos, tais que 
\[a + b + \mdc(a,b) = 33?\]

\textcolor{blue}{\bf(6)} Considere todos os inteiros com nove algarismos distintos (em base decimal), todos diferentes de 0. Encontre o mdc de todos eles.
\newline\newline
\textcolor{blue}{\bf(7)} Encontre o mdc dos números $2n+13$ e $n+7.$
\newline\newline
\textcolor{blue}{\bf(8)} Prove que a fração $\dfrac{12n+1}{30n+2}$ é irredutível para qualquer inteiro $n.$
\newline\newline
\textcolor{blue}{\bf(9)} Encontre $\mdc(1 \dots 111,11\dots 11),$ onde a representação decimal do primeiro número tem cem algarismos igual a $1$ e o segundo número tem sessenta.
\newline
\newline
\textcolor{blue}{\bf(10)} Encontre todos os pares ordenados $(a,b),$ com $a$ e $b$ inteiros positivos, tais que
$\mdc(a,b) = 15$ e $\mmc(a,b) = 300.$ 

\textcolor{blue}{\bf(11)} Quatro números inteiros positivos $a < b < c < d$ são tais que o mdc entre quaisquer
dois deles é maior do que $1,$ mas o mdc entre todos eles é 1, ou seja, $\mdc(a,b,c,d) = 1.$  Qual é o menor valor possível para $d$?
\newline\newline
\textcolor{blue}{\bf(12)} Qual é o maior valor possível do mdc de dois números distintos pertencentes ao conjunto
$1,2,3, \ldots ,2011?$
\newline \newline
%https://math.stackexchange.com/questions/2225355/how-to-prove-gcda-b-cdot-gcdc-d-gcdac-ad-bc-bd?rq=1
\textcolor{blue}{\bf(13) $\varheart$} Sejam $r, a, b, c, d$ inteiros positivos. Mostre que
\begin{tasks}[counter-format={(tsk[a])},label-width=3.6ex, label-format = {\bfseries}, column-sep = {0pt}](1)
\task[\textcolor{Floresta}{$\negrito{(a)} $}] $\mdc(ar, br) = r \mdc(a,b);$
\task[\textcolor{Floresta}{$\negrito{(b)} $}] $\mdc(a,b) \cdot \mdc(c,d) = \mdc(ac, ad, bc, bd).$
\end{tasks}

%https://math.stackexchange.com/questions/977617/if-gcd-gcda-b-gcda-c-1-then-gcda-bc-gcda-b-cdot-gcda?rq=1
\textcolor{blue}{\bf(14) $\varheart$} Sejam $a, b$ e $c$ inteiros. Prove que se $\mdc(a, b)$ e $\mdc(a, c)$ são coprimos, ou seja,
\[
\mdc(\mdc(a,b), \mdc(a,c)) = 1,
\] então
\[\mdc(a, bc) = \mdc(a, b) \cdot \mdc(a, c).\]
\textsf{Dica:} Utilize a Relação de Bézout.
\newline\newline
\textcolor{blue}{\bf(15) $\varheart$} Um empreiteiro deseja construir um prédio em um terreno retangular de dimensões $216$ m por $414$ m. Para isso deverá cercá-lo com estacas. Se ele colocar uma estaca em cada canto do terreno e utilizar sempre a mesma distância entre duas estacas consecutivas, qual será a quantidade mínima de estacas a serem utilizadas?
\newline\newline
\textcolor{blue}{\bf(16) $\varheart$} Encontre o máximo divisor comum $d$ de $1819$ e $3587$ e determine inteiros $r,s$ tais que
\[1819r + 3587s = d\]

\textcolor{blue}{\bf(17) $\varheart$} Sejam $a$ e $b$ dois números inteiros primos entre si. Mostre que
\begin{tasks}[counter-format={(tsk[a])},label-width=3.6ex, label-format = {\bfseries}, column-sep = {0pt}](1)
\task[\textcolor{Floresta}{$\negrito{(a)} $}] $\mdc(a+b, a^2 - ab  +b^2) = 1 \text{ ou } 3;$
\task[\textcolor{Floresta}{$\negrito{(b)} $}] $\mdc(a + b,a^2 + b^2)= 1 \text{ ou } 2.$
\end{tasks}
Investigue os possíveis valores de $\mdc(a^n + b^n, a^m + b^m),$ onde $n, m \ge 1.$
\newline\newline
\textcolor{blue}{\bf(18) $\varheart$} Uma certa tinta pode ser comprada em galões de $18\ell$ ou em latas de $3 \ell.$ Precisa-se de $250 \ell$ dessa tinta. De quantas maneiras se pode comprar latas e galões para que a quantidade de sobra seja mínima?
\newline\newline
\textcolor{blue}{\bf(19) $\varheart$} Encontre todos os inteiros positivos $a$ tais que
\[
\begin{cases}
\mmc(120,a) = 360 \\
\mdc(450, a) = 90
\end{cases}
\]
\textcolor{blue}{\bf(20) $\varheart$} Resolva em $\mathbb{Z}$ os seguintes sistemas:
\begin{tasks}[counter-format={(tsk[a])},label-width=3.6ex, label-format = {\bfseries}, column-sep = {0pt}](3)
\task[\textcolor{Floresta}{$\negrito{(a)} $}] $\begin{cases}
\mdc(x,y) = 12 \\
\mmc(x,y) = 168
\end{cases}$
\task[\textcolor{Floresta}{$\negrito{(b)} $}] $\begin{cases}
\mdc(x,y) = 20 \\
\mmc(x,y) = 420
\end{cases}$
\task[\textcolor{Floresta}{$\negrito{(c)} $}] $\begin{cases}
\mdc(x,y) = 21 \\
\mmc(x,y) = 1134
\end{cases}$
\end{tasks}



Exercícios marcados com $\varheart$ são extras.
\end{document}
SetClassGroupBounds("GRH"); 
K := QuadraticField(9);
ClassNumber(K);
Q := PolynomialRing(GF(2), 2);
Q;
K := QuadraticField();
G := GaloisGroup(K);
G;
\begin{CJK}{UTF8}{min}
露の世は 露の世ながら さりながら当時では老人と呼べる50歳代半ばでようやく授かったわが子への愛とその突然死を伝えた「露の世」のくだりは、その日記体句文集「おらが春」のクライマックスとなっています。

5月には数え二歳の誕生を迎えて詠んだ句に、
「這へ笑へ二つになるぞけさからは」と喜びを謳歌したばかり。

それが、翌6月にはもはや草葉の陰へと、その露の朝日に立ちどころに消えるごとく、儚くも身罷ってしまったとは。

人の世は朝露の如く無常なのだと、悔みを述べ慰問するあの人、この人。さは「さりながら」…それは確かにそうなのだけれども。
いかに「あきらめ顔しても、思い切りがたきは、恩愛のきづな也けり。」と、わが心中は耐えきれず切々と泣き崩れるばかり。わが子「さと」女を思う「大切」はやがて「あなた任せ」の境地へと通じていくものでしょう。
\end{CJK}